\documentclass[12pt]{article}

\usepackage[utf8]{inputenc}
\usepackage{fullpage}
\usepackage{amsmath,amsthm,amsfonts,amssymb}
\usepackage[capitalize,nameinlink]{cleveref}
\usepackage{mathtools}
\usepackage{xcolor}
\usepackage{physics}
\usepackage[shortlabels]{enumitem}

\newcommand{\eps}{\varepsilon}
\newcommand{\bitset}{\{0,1\}}
\newcommand{\fbitset}{\{1,-1\}}
\newcommand{\tritset}{\{0,1,\bot\}}
\newcommand{\E}[2]{\mathbb E_{#1}\left[#2\right]}
\renewcommand{\P}{\mathbb P}
\newcommand{\F}{\mathbb F}
\newcommand{\N}{\mathbb N}
\newcommand{\Z}{\mathbb Z}
\newcommand{\R}{\mathbb R}
\newcommand{\poly}{\mathrm{poly}}
\newcommand{\polylog}{\mathrm{polylog}}
\newcommand{\supp}{\mathrm{supp}}
\newcommand{\Inf}{\mathbf{Inf}}
\newcommand{\Var}[2]{\mathbf{Var}_{#1}\left[#2\right]}
\newcommand{\Stab}{\mathbf{Stab}}
\newcommand{\set}[1]{\{#1\}}
\newcommand{\eqn}[1]{\begin{equation*}#1\end{equation*}}
\newcommand{\hf}{\widehat{f}}
\newcommand{\sumx}{\sum_{x \in \bitset^n}}
\newcommand{\sumxf}{\sum_{x \in \fbitset^n}}
\newcommand{\sumS}{\sum_{S \subseteq [n]}}
\renewcommand{\ip}[1]{\left\langle#1\right\rangle}

\begin{document}
\section{Chapter 1}

\subsection{}

\begin{enumerate}[(a)]
    \item \begin{align*}
		     \widehat{\min}_2(00) = \ip{\chi_{00}, f} &= \frac{1}{4}(f(1,1) + f(1,-1) + f(-1,1) + f(-1,-1))\\
		     &= -\frac{1}{2}
	      \end{align*}

	      \begin{align*}
		     \widehat{\min}_2(01) = \ip{\chi_{01}, f} &= \frac{1}{4}(f(1,1) + f(1,-1) - f(-1,1) - f(-1,-1))\\
		     &= \frac{1}{2}
	      \end{align*}

	      \begin{align*}
		     \widehat{\min}_2(10) = \ip{\chi_{10}, f} &= \frac{1}{4}(f(1,1) - f(1,-1) + f(-1,1) - f(-1,-1))\\
		     &= \frac{1}{2}
	      \end{align*}

	      \begin{align*}
		     \widehat{\min}_2(11) = \ip{\chi_{11}, f} &= \frac{1}{4}(f(1,1) - f(1,-1) - f(-1,1) + f(-1,-1))\\
		     &= \frac{1}{2}
	      \end{align*}

	      \begin{align*}
	      	{\min}_2(x,y) &= \frac{1}{2}(-\chi_{00} + \chi_{01} + \chi_{10} + \chi_{11})(x,y)\\
	      	&= \frac{1}{2}(-1 + y + x + xy)
	      \end{align*}

    \item \begin{align*}
		     \widehat{\min}_3(000) = \ip{\chi_{000}, f} &= \frac{1}{8}(f(1,1,1) + f(1,1,-1) + f(1,-1,1) + f(1,-1,-1)\\
		     &\qquad + f(-1,1,1) + f(-1,1,-1) + f(-1,-1,1) + f(-1,-1,-1))\\
		     &= -\frac{3}{4}
	      \end{align*}
	      \eqn{\widehat{\min}_3(001) = \widehat{\min}_3(010) = \widehat{\min}_3(100) = \frac{1}{4}}
	      \eqn{\widehat{\min}_3(011) = \widehat{\min}_3(101) = \widehat{\min}_3(110) = \frac{1}{4}}
	      \eqn{\widehat{\min}_3(111) = \frac{1}{4}}
	      \begin{align*}
	      	{\min}_3(x,y,z) &= \frac{1}{4}(-3\chi_{000} + \chi_{001} + \chi_{010} + \chi_{100} + \chi_{011} + \chi_{101} + \chi_{110} + \chi_{111})(x,y,z)\\
	      	&= \frac{1}{4}(-3 + z + y + x + yz + xz + xy + xyz)
	      \end{align*}
	      \eqn{{\max}_3(x,y,z) = - {\min}_3(-x,-y,-z) = \frac{1}{4}(3 + z + y + x - yz - xz - xy + xyz)}

    \item $1_{\set{a}}(x) = \frac{1}{2^n}\sumS \chi_S(a) \chi_S(x) = \frac{1}{2^n}\sumS \chi_S(a+x)$

    \item $\varphi_{\set{a}}(x) = 2^n 1_{\set{a}}(x) = \sumS \chi_S(a+x)$

    \item $\varphi_{\set{a + e_i}} = 2^{n-1} (1_{\set{a}} + 1_{\set{e_i}}) = \frac{1}{2}\sumS (\chi_S(a) + \chi_S(e_i)) \chi_S(x)$

    \item As the density corresponding to coordinate $i$ is $2^{n-1}\left((1 + \rho) 1_{\set{e_i}} + (1 - \rho)1_{\set{-e_i}}\right)$, we have
	      \begin{align*}
	      	\widehat{\phi}(S) &= \frac{1}{2^n}\prod_{i=1}^n((1+\rho)\chi_S(e_i) + (1-\rho)\chi_S(-e_i))\\
	      	&= \frac{1}{2^n}\prod_{i=1}^n((1+\rho)s_{e_i} - (1-\rho) s_{-e_i})
	      \end{align*}

	      \item \begin{align*}
	      		  \ip{\text{IP}_{2n}, \chi_{a,b}} &= \frac{1}{2^{2n}}\sum_{x, y \in \bitset^n} (-1)^{x \cdot y} (-1)^{a \cdot x + b \cdot y}\\
	      		  &= \frac{1}{2^{2n}}\sumx (-1)^{a \cdot x} \sum_{y \in \bitset^n} (-1)^{y \cdot (x + b)}.
	      		\end{align*}
	     As the inner sum is 0 if $x \neq b$ and $2^n$ otherwise,
	     \eqn{\text{IP}_{2n}(x,y) = \frac{1}{2^n}\sum_{a, b \in \bitset^n} (-1)^{a \cdot b} \chi_{a,b}(x,y)}

    \item As $\ip{\text{EQ}_n,\chi_S} = 2^{-n}(\chi_S(0^n) + \chi_S(1^n))$,
	      \eqn{\text{EQ}_n(x) = 2^{n-1}\sum_{\substack{S \subseteq [n]\\\abs{S} \text{ even}}} \chi_S(x)}
	      (We take the domain to be $\F_2^n$.)

    \item We have $\text{NEQ}_n(x) = 1 + \text{EQ}_n$, so
	      \eqn{\text{NEQ}_n(x) = 2^{n-1}\sum_{\substack{S \subseteq [n]\\\abs{S} \text{ odd}}} \chi_S(x)}

    \item As $\widehat{\text{Sel}}(001) = \widehat{\text{Sel}}(010) = \widehat{\text{Sel}}(101) = - \widehat{\text{Sel}}(110) = \frac{1}{2}$,
		    \eqn{\text{Sel}(x) = \frac{1}{2}(x_2 + x_3 + x_1 x_3 - x_1 x_2)}

    \item Since $\ip{\text{mod}_3, \chi_{S}} = \frac1{8}(\chi_{S}(000) + \chi_{S}(111))$,
	      \eqn{\text{mod}_3 = \frac1{4} \left(1 + \chi_{\set{1,2}} + \chi_{\set{1,3}} + \chi_{\set{2,3}}\right)}

    \item

    \item

    \item

    \item
\end{enumerate}
	
\subsection{}
 $2n+2$, corresponding to the constant and dictator/antidictator functions.
	
\subsection{}
For every $S \subseteq [n]$, we have
\eqn{\ip{f, \chi_S} = \frac1{2^n} \sumx f(x) \chi_S(x) = \sum_{f(x) = 1} \chi_S(x),}
and since $\chi_S(x) \in \fbitset$ and $\abs{\set{x \in \bitset^n : f(x) = 1}}$ is odd, the sum is nonzero.

\subsection{}

\subsection{}

\subsection{}
%If $f = \sumS \hf_1(S) \chi_S = \sumS \hf_2(S) \chi_S$, then Parseval implies $\sumS \hf_1(S) \hf_2(S) = 1$. Since $\abs{\hf_1(S) \hf_2(S)} \leq 1$, equality

\subsection{}
For all $S \subseteq [n]$, we have $\hat{\mathbf{f}}(S) = \ip{\chi_S,\mathbf{f}} = \frac1{2^n}\sumxf\mathbf{f}(x) \chi_S(x)$, so 
\eqn{\E{\mathbf{f}}{\hat{\mathbf{f}}(S)} = \E{\mathbf{f}}{\frac1{2^n}\sumxf \mathbf{f}(x) \chi_S(x)} = \frac1{2^n}\sumxf\chi_S(x) \E{\mathbf{f}}{\mathbf{f}(x)} = 0.}
Moreover,
\begin{align*}
    \Var{}{\mathbf{f}(S)} = \E{\mathbf{f}}{\mathbf{\hat{f}}(S)^2} &= \frac1{2^{2n}}\E{\mathbf{f}}{2^n + \sumx \sum_{\substack{y \in \fbitset^n\\ y \neq x}} \chi_S(x) \chi_S(y) \mathbf{f}(x) \mathbf{f}(y)}\\
    &= \frac1{2^n} + \sumx \sum_{\substack{y \in \fbitset^n\\ y \neq x}} \chi_S(x) \chi_S(y) \E{\mathbf{f}}{\mathbf{f}(x) \mathbf{f}(y)}\\
    &= \frac1{2^n}.
\end{align*}

\subsection{}

\subsection{}

\subsection{}

\subsection{}

\subsection{}

\subsection{}

\subsection{}

\subsection{}

\subsection{}

\subsection{}

\subsection{}

\subsection{}

\subsection{}

\subsection{}

\subsection{}

\subsection{}

\subsection{}

\subsection{}

\subsection{}

\subsection{}

\subsection{}

\subsection{}
\begin{enumerate}[(a)]
    \item ($\Longrightarrow$) $f(x) + f(y) + f(z) = a \cdot (x + y + z) + 3b = a \cdot (x + y + z) + b = f(x+y+z)$.
    
    ($\Longleftarrow$) Let $b \coloneqq f(0)$ and $a \coloneqq (f(e_1) - b, f(e_2) - b, \ldots, f(e_n) - b)$. Then,
    \begin{align*}
    f(x) = f\left(\sum_{i = 1}^n x_i \cdot e_i\right) &= f(x_i e_i) + f(0) + f\left(\sum_{i = 2}^n x_i e_i\right)\\
    &= f(x_n e_n) + \sum_{i = 1}^{n-1} f(x_i e_i) + f(0)\\
    &= f(0) + \sum_{i = 1}^{n} f(x_i e_i) + f(0)\\
    &= a \cdot x + b,
    \end{align*}
    since $f(x_i e_i) = \left\{\begin{array}{lll}f(0) &= b, &\text{if } x_i = 0,\\f(e_i) + f(0) &= a_i, &\text{if } x_i = 1\end{array}\right. = a_i x_i$.
    
    \item We have
    \begin{align*}
        \E{x,y,z}{f(x)f(y)f(z)f(x+y+z)} &= \E{x,y}{f(x)f(y)\E{z}{f(z)f(x+y+z)}}\\
        &= \E{x,y}{f(x) f(y) \cdot (f * f)(x+y)}\\
        &= \E{x}{f(x) \E{y}{f(y) (f * f)(x+y)}}\\
        &= \E{x}{f(x) \cdot (f * f * f)(x)}\\
        &= \ip{f, f * f * f}\\
        &= \ip{\hat{f}, \widehat{f * f * f}}\\
        &= \sumS \hat{f}(S)^4.
    \end{align*}
    
    \item Consider the test that samples $x, y, z \sim \bitset^n$ uniformly and independently, queries $f(x)$, $f(y)$, $f(z)$ and $f(x + y + z)$, accepting if and only if $f(x+y+z) = f(x) + f(y) + f(z)$. Encoding the output of $f$ by $\fbitset$, the expression$\frac1{2} + \frac1{2} f(x) f(y) f(z) f(x+y+z)$ is the indicator of the test's acceptance. Then,
    \eqn{\P[\text{test accepts}] = \frac1{2} + \frac1{2} \sumS \hat{f}(S)^4 \geq 1 - \eps}
    implies $1 - 2\text{dist}(f, \chi_T) = \hat{f}(T) \geq \hat{f}(T)^2  = \hat{f}(T)^2 \cdot \sumS \hat{f}(S)^2 \geq \sumS \hat{f}(S)^4 \geq 1 - 2\eps$ for some $T \subseteq [n]$. Thus, $\text{dist}(f, \chi_T) \leq \eps$.
    
    \item It is clear that if $f(x) + f(y) + f(z) = f(x + y + z)$ holds for all $x,y,z$, then it does in particular when $z = 0$. Conversely, if we are only ensured of $f(x) + f(y) + f(0) = f(x + y)$, then $f(x + y + z) = f(x + y) + f(z) + f(0) = f(x) + f(y) + f(z) + 2f(0) = f(x) + f(y) + f(z)$; therefore, by (a), $f$ is affine if and only if $f(x+y) = f(x) + f(y) + f(0)$ for all $x,y$.
    
    The test samples $x,y \sim \bitset^n$ uniformly and independently, queries $f(x)$, $f(y)$, $f(0)$ and $f(x+y)$, accepting when $f(x+y) = f(x) + f(y) + f(0)$. Shifting to $\fbitset$ notation for the codomain of $f$, the indicator of acceptance is $\frac1{2} + \frac{f(0)}{2} \cdot f(x) f(y) f(x+y)$ and the same analysis of the BLR test works (the factor $f(0)$ disappears when taking absolute values).
\end{enumerate}

\subsection{}
\begin{enumerate}[(a)]
    \item We have $f^\pi(x) = f(x^\pi) = \sumS \hat{f}(S) \chi_S(x^\pi) = \sumS \hat{f}(S) \chi_{\pi^{-1}(S)}(x)$, so
    \eqn{f^\pi = \sumS \hat{f}(S) \chi_{\pi^{-1}(S)} = \sumS \hat{f}\left(\pi^{-1}(S)\right) \chi_S}
    and thus $\widehat{f^\pi}(S) = \hat{f}(\pi^{-1}(S))$ for all $S$.
    
    \item Indeed: for any $\pi, \sigma \in P_n$, we have
    \eqn{f^\pi - f^\sigma = \sum_{i = 0}^k \sum_{\abs{S} = k} \left(\widehat{f^\pi}(S) - \widehat{f^\sigma}(S)\right)\chi_S = \sum_{i = 0}^k \sum_{\abs{S} = k} \left(\hat{f}\left(\pi^{-1}(S)\right) - \hat{f}\left(\sigma^{-1}(S)\right)\right)\chi_S,}
    and the fact that $\pi, \sigma \in P_k$ for all $k$ implies $\hat{f}\left(\pi^{-1}(S)\right) \geq \hat{f}\left(\sigma^{-1}(S)\right)$ as well as $\hat{f}\left(\sigma^{-1}(S)\right) \geq \hat{f}\left(\pi^{-1}(S)\right)$ for all $S$; therefore the right-hand side of the expression is 0, so that $f^\pi = f^\sigma$.
    
    \item Condition (i) holds by definition, since $\text{can}(f) = f^\pi$ for some permutation $\pi$. Condition (ii) holds because $g = f^\pi$ and $\sigma \in P_n$ with respect to $g$ if and only if $\sigma \circ \pi \in P_n$ with respect to $f$. Thus, $\text{can}(g) = g^\sigma = \left(f^\pi\right)^\sigma = f^{\sigma \circ \pi} = \text{can}(f)$.
    
    \item For each $i \in [n]$, computing $\hat{f}(i)$ takes $O(2^n)$ time, for a total complexity of $O(n 2^n) = \tilde{O}(2^n)$. Then, since under this condition there exists a single permutation $\pi \in S_n$ that makes the sequence $\left(\hat{f}(i)\right)_{i \in [n]}$ maximal (which can be obtained by sorting $n$ numbers, whose time is dominated by the computation of the $\hat{f}(i)$), the algorithm then outputs $f^\pi$.
    
    \item
\end{enumerate}

\subsection{}

\subsection{}

\subsection{}

\subsection{}

\subsection{}
\end{document}
